%!TEX TS-program = xelatex
\documentclass[]{friggeri-cv}
\addbibresource{bibliography.bib}

\begin{document}
\header{Paweł}{Kaźmierzewski}
       {Inżynier oprogramowania wbudowanego}


% In the aside, each new line forces a line break
\begin{aside}
  \section{o mnie}
    gen.R.Abrahama 2A 18
    03-982 Warszawa
    Poland
    ~
    \href{mailto:pawel.kazmierzewski@gmail.com}{pawel.kazmierzewski@gmail.com}
  \section{języki}
    polski  ojczysty
    angielki płynnie
  \section{programowanie}
  {\color{red} $\varheartsuit$} {\bf C/C++}
    Matlab, Java
    Bash, Python
\end{aside}

\section{zainteresowania}

gitara, gry komputerowe, sport - crossfit i bieganie, terrarystyka, majsterkowanie, książki 

\section{wykształcenie}

\begin{entrylist}
  \entry
  {2012–2013}
  {Studia podyplomowe {\normalfont Inżynieria oprogramowania}}
  {Politechnika Warszawska}
  {}
  \entry
  {2005–2010}
  {Magister inżynier
    {\normalfont Mechatronika\\
    Specjalizacja: Awionika i uzbrojenie lotnicze}}
  {Wojskowa Akademia Techniczna}
  {\emph{System automatycznego startu i lądowania bezpilotowej platformy latającej}}
  \entry
    {2002–2005}
    {Liceum}
    {XXXIX Liceum Ogólnokształcące im. Lotnictwa Polskiego}
    {Profil matematyczno-fizyczny}
\end{entrylist}

\section{doświadczenie}

\begin{entrylist}
  \entry
    {02/2014–teraz}
    {WB Electronics S.A.}
    {Starszy Inżynier Oprogramowania}
    {\emph{Opracowanie modułu nawigacji intercyjnej}}
  \entry
    {08/2013–01/2014}
    {Samsung Electronics Poland Sp. z o.o}
    {Niezależny kontraktor}
    {\emph{Rozwój średniej warstwy systemu}}
  \entry
    {05/2013–07/2013}
    {Cyfrowy Polsat S.A.}
    {Inżynier Oprogramowania}
    {\emph{Rozwój niskopoziomowego oprogramowania}}
  \entry
    {10/2010–04/2013}
    {WB Electronics S.A.}
    {Projektant Systemów Autonomicznych}
    {\emph{Projektowanie i rozwój bezpilotowego statku powietrznego}}
  \entry
    {07/2010–09/2010}
    {WB Electronics S.A.}
    {Stażysta}
    {\emph{Obsługa naziemna latającego drona}}
  \entry
    {07/2009–08/2009}
    {Takom Sp. z o.o}
    {Praktyka dyplomowa}
    {\emph{Prace remontowe w przepompowni ciepłej wody w Warszawie}}
\end{entrylist}

\section{kursy i szkolenia}

\begin{entrylist}
  \entry
    {2011}
    {Podstawy wykorzystania systemu Linux w systemach wbudowanych}
    {}
    {}
  \entry
    {2010}
    {AutoCAD 2010 kurs średnio zaawansowany}
    {}
    {}
  \entry
    {2008–2009}
    {Ukończone 4 semestry CISCO CCNA}
    {}
    {}
\end{entrylist}

\section{dodatkowe umiejętności}

\begin{description}
  \item [Oprogramowanie inżynierskie] Znajomość pakietu Matlab, a w szczególności modułu Simulink; LabVIEW; Protel; AutoCAD
  \item [Umiejętności praktyczne] Lutowanie elementów przewlekanych oraz SMD; doświadczenie w posługiwaniu się\
    narzędziami warsztatowymi; umiejętność czytania oraz interpretowania schematów elektrycznych i mechanicznych
  \item [Użytkowanie komputera] Znajomość systemu Linux oraz Windows na poziomie zaawansowanym; obsługa wielu pakietów biurowych
  \item [Umiejętności specjaliztyczne] Implementacja złożonych algorytmów, a także ich modyfikacje\
    (praktyczne ich użycie w precyzyjnym sprzęcie latającym);\ 
    doświadczenie w obsłudze wielu czujników i urządzeń z wykorzystaniem popularnych interfejsów mikrokontrolerowych\
    (UART, SPI, {\normalfont I\(^{2}\)C}); znajomość architektury ARM (arm7, Cortex-M, Cortex-A)
  \item [Inne] Prawo jazdy kategorii B
\end{description}
\null
\vfill
\emph{Wyrażam zgodę na przetwarzanie moich danych osobowych zawartych\
 w przesłanym CV dla potrzeb niezbędnych w procesie rekrutacji,\
  zgodnie z ustawą z dnia 29.08.1997 o Ochronie danych Osobowych\
 (Dz.U.Nr.133 poz.883)}
%\begin{flushbottom}
  %asdfasdfasdf
%\end{flushbottom}
%%% This piece of code has been commented by Karol Kozioł due to biblatex errors. 
% 
%\printbibsection{article}{article in peer-reviewed journal}
%\begin{refsection}
%  \nocite{*}
%  \printbibliography[sorting=chronological, type=inproceedings, title={international peer-reviewed conferences/proceedings}, notkeyword={france}, heading=subbibliography]
%\end{refsection}
%\begin{refsection}
%  \nocite{*}
%  \printbibliography[sorting=chronological, type=inproceedings, title={local peer-reviewed conferences/proceedings}, keyword={france}, heading=subbibliography]
%\end{refsection}
%\printbibsection{misc}{other publications}
%\printbibsection{report}{research reports}

\end{document}
